%%%% Martin Rotter, Jiří Zehnula, Radek Janoštík - FJAA (zápisky z přednášek)
%%%%
%%%% kompilováno přes cslatex --> dvips --> (případně) ps2pdf
%%%%
%%%%%%%%%%%%%%%%%%%%%%%%%%%%%%%%%%%%%%%%%%%%%%%%%%%%%%%%%%%%%%%%%%%%%%%%%%%%%%%%%
%%%%%%%%%%%%%%%%%%%%%%%%%%%%%%%%%%%%%%%%%%%%%%%%%%%%%%%%%%%%%%%%%%%%%%%%%%%%%%%%%

\documentclass[10pt, a4paper, titlepage]{article}

\usepackage{czech}						% napsáno česky
\usepackage[utf8]{inputenc}				% napsáno v UTF-8
\usepackage[tables,figures]{upreport}	% UPOL styl
%\usepackage{upstyles}
\usepackage{a4wide}
\usepackage{amsmath}
\usepackage{amssymb}
\usepackage{color}
\usepackage{amsthm}
\usepackage{vaucanson-g}				% makro pro kreslení grafů a automatů
\usepackage{alltt}

%\renewcommand{\familydefault}{\sfdefault}

\setlength{\parindent}{0pt}
\setlength{\parskip}{1ex plus 0.5ex minus 0.2ex}

\title{KMI/DATA1  -- Databáze 1}
\author{Tomáš Kukučka (KukuckaT@gmail.com)}
\date{\today}

\docinfo{Tomáš Kukučka}{KMI/DATA1  -- Databáze 1}

\newtheoremstyle{note}
{15pt}
{15pt}
{}
{}
{\bfseries}
{:}
{.5em}
{}
\theoremstyle{note}
\newtheorem{dukaz}{Důkaz}
\newtheorem{veta}{Věta}
\newtheorem{definice}{Definice}
\newtheorem{priklad}{Příklad}
\newtheorem{poznamka}{Poznámka}
\newtheorem{dusledek}{Důsledek}


\abstract{
Tento dokument je pouze přepisem zápisků a poznámek
z přednášek předmětu KMI/DATA1. Přednášel \link{doc. Vilém Vychodil PhD}{http://vychodil.inf.upol.cz/}.
}
\pagestyle{empty}
\makeindex

\begin{document}
\maketitle

\section{Přehled databázových systémů}
\textbf{Databázový systém:}
\begin{itemize}
	\item teorie - návrh systému pro organizaci dat
	\begin{itemize}
		\item formální metody
		\item algoritmy (\uv{Aby to celé bylo rozumné}) - na vrstvě logické, na vrstvě fyzické (jak uložit na disk,...)
	\end{itemize}
	\item software
	\begin{itemize}
		\item \uv{produkt} (implementuje výše řešený model)
	\end{itemize}
\end{itemize}

\textbf{DBMS} (česky SŘBD)- database management system (systém řízení báze dat)
\begin{itemize}	
	\item databáze = kolekce dat obhospodařovaná DBMS
	\item poskytuje:
	\begin{itemize}
		\item perzistentní ("když vypnu a zapnu počítač, data zůstanou") uložení dat
		\item rozhraní pro uživatele (programy)
		\item transakční zpracování dat (\uv{pokud já chci provést sérii modifikací dat, tak chci zaručit, aby série provádějící
			 modifikaci, aby proběhla celá}) - některé DB systémy to nedělají
	\end{itemize}
\end{itemize}		 


\subsection{Historie}

\textbf{Filesystem} (souborový systém)
\begin{itemize}
	\item aplikace si to řešily po svém (zapisování do souboru)
	\item klady
	\begin{itemize}
		\item dá se poměrně snadno naprogramovat
	\end{itemize}		 
	\item zápory
	\begin{itemize}
		\item programátor se zabývá fyzickou reprezentací dat
		\item nechtěná redundance dat
		\item praktická "nemožnost" sdílení mezi síťovými aplikacemi
	\end{itemize}
\end{itemize}

\textbf{COBOL} 1959
\begin{itemize}
	\item první rozumná verze 1968
	\item poslední standart 2002 (poprvé uživatelsky definované funkce)
	\item před COBOLEM FLOW-MATIC (vytvořila Grace Hopper)
	\item považuje se za \uv{antijazyk}
	\item Dijsktra: \uv{The use of COBOL cripples the mind; its teaching should, therefore, be regarded as a criminal offense.}
\end{itemize}

\textbf{Síťový databázový model}
\begin{itemize}
	\item Charles Bachtman
	\item překonané paradigma (pozor na objekty)
	\item CODASYL - standardizovali moc dlouho a byl překonán relačním modelem
	\item databáze se skládají z:
	\begin{itemize}
		\item záznamů (record)
		\begin{itemize}
			\item dán typem záznamů
			\item typ deklaruje hodnoty (v záznamech mají hodnoty)
		\end{itemize}
		\item odkazy (link)
		\begin{itemize}
			\item reprezentují vztahy mezi záznamy - analogie je asociace (ukazatele)
		\end{itemize}
	\end{itemize}
	\item dotazování je dřina, vytváření dat je dřina
	\item můžou být rychlé, pokud jsou dotazy jednoduché
	\item zakreslují se síťovým diagramem
	\item  obr 1, 2, 3
\end{itemize}


\textbf{Relační databázový model} (70. léta)
\begin{itemize}
	\item klady
	\begin{itemize}
		\item používá se
		\item jeden typ objektu = relace
	\end{itemize}
	\item autor Egar F. Codd - napsal knihu:  A relational model of data for large shared data banks
\end{itemize}

\textbf{Objektový databázový model}
\begin{itemize}
	\item není shoda, co by měla být objektová databáze
	\item poprvé v Lispu - balík STATICE
	\item v současné době v Common Lispu balík Elephant
\end{itemize}

\textbf{Nerelační databáze (90. léta)}
\begin{itemize}
	\item MongoDB
	\item Berkeley DB
	\item někdy označovány jako NoSQL
\end{itemize}

%\newpage
\renewcommand{\indexcolumns}{3}
\printindex
\end{document}